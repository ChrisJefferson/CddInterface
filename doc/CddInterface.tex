% generated by GAPDoc2LaTeX from XML source (Frank Luebeck)
\documentclass[a4paper,11pt]{report}

\usepackage{a4wide}
\sloppy
\pagestyle{myheadings}
\usepackage{amssymb}
\usepackage[utf8]{inputenc}
\usepackage{makeidx}
\makeindex
\usepackage{color}
\definecolor{FireBrick}{rgb}{0.5812,0.0074,0.0083}
\definecolor{RoyalBlue}{rgb}{0.0236,0.0894,0.6179}
\definecolor{RoyalGreen}{rgb}{0.0236,0.6179,0.0894}
\definecolor{RoyalRed}{rgb}{0.6179,0.0236,0.0894}
\definecolor{LightBlue}{rgb}{0.8544,0.9511,1.0000}
\definecolor{Black}{rgb}{0.0,0.0,0.0}

\definecolor{linkColor}{rgb}{0.0,0.0,0.554}
\definecolor{citeColor}{rgb}{0.0,0.0,0.554}
\definecolor{fileColor}{rgb}{0.0,0.0,0.554}
\definecolor{urlColor}{rgb}{0.0,0.0,0.554}
\definecolor{promptColor}{rgb}{0.0,0.0,0.589}
\definecolor{brkpromptColor}{rgb}{0.589,0.0,0.0}
\definecolor{gapinputColor}{rgb}{0.589,0.0,0.0}
\definecolor{gapoutputColor}{rgb}{0.0,0.0,0.0}

%%  for a long time these were red and blue by default,
%%  now black, but keep variables to overwrite
\definecolor{FuncColor}{rgb}{0.0,0.0,0.0}
%% strange name because of pdflatex bug:
\definecolor{Chapter }{rgb}{0.0,0.0,0.0}
\definecolor{DarkOlive}{rgb}{0.1047,0.2412,0.0064}


\usepackage{fancyvrb}

\usepackage{mathptmx,helvet}
\usepackage[T1]{fontenc}
\usepackage{textcomp}


\usepackage[
            pdftex=true,
            bookmarks=true,        
            a4paper=true,
            pdftitle={Written with GAPDoc},
            pdfcreator={LaTeX with hyperref package / GAPDoc},
            colorlinks=true,
            backref=page,
            breaklinks=true,
            linkcolor=linkColor,
            citecolor=citeColor,
            filecolor=fileColor,
            urlcolor=urlColor,
            pdfpagemode={UseNone}, 
           ]{hyperref}

\newcommand{\maintitlesize}{\fontsize{50}{55}\selectfont}

% write page numbers to a .pnr log file for online help
\newwrite\pagenrlog
\immediate\openout\pagenrlog =\jobname.pnr
\immediate\write\pagenrlog{PAGENRS := [}
\newcommand{\logpage}[1]{\protect\write\pagenrlog{#1, \thepage,}}
%% were never documented, give conflicts with some additional packages

\newcommand{\GAP}{\textsf{GAP}}

%% nicer description environments, allows long labels
\usepackage{enumitem}
\setdescription{style=nextline}

%% depth of toc
\setcounter{tocdepth}{1}





%% command for ColorPrompt style examples
\newcommand{\gapprompt}[1]{\color{promptColor}{\bfseries #1}}
\newcommand{\gapbrkprompt}[1]{\color{brkpromptColor}{\bfseries #1}}
\newcommand{\gapinput}[1]{\color{gapinputColor}{#1}}


\begin{document}

\logpage{[ 0, 0, 0 ]}
\begin{titlepage}
\mbox{}\vfill

\begin{center}{\maintitlesize \textbf{ CddInterface \mbox{}}}\\
\vfill

\hypersetup{pdftitle= CddInterface }
\markright{\scriptsize \mbox{}\hfill  CddInterface  \hfill\mbox{}}
{\Huge \textbf{ Gap interface to Cdd package \mbox{}}}\\
\vfill

{\Huge  0.1 \mbox{}}\\[1cm]
{ 18/11/2015 \mbox{}}\\[1cm]
\mbox{}\\[2cm]
{\Large \textbf{ Kamal Saleh\\
    \mbox{}}}\\
\hypersetup{pdfauthor= Kamal Saleh\\
    }
\end{center}\vfill

\mbox{}\\
{\mbox{}\\
\small \noindent \textbf{ Kamal Saleh\\
    }  Email: \href{mailto://kamal.saleh@rwth-aachen.de} {\texttt{kamal.saleh@rwth-aachen.de}}\\
  Homepage: \href{} {\texttt{}}\\
  Address: \begin{minipage}[t]{8cm}\noindent
 Templergraben \\
 \end{minipage}
}\\
\end{titlepage}

\newpage\setcounter{page}{2}
\newpage

\def\contentsname{Contents\logpage{[ 0, 0, 1 ]}}

\tableofcontents
\newpage

 \index{\textsf{CddInterface}}     
\chapter{\textcolor{Chapter }{Functions and Methods}}\label{Chapter_Functions_and_Methods}
\logpage{[ 1, 0, 0 ]}
\hyperdef{L}{X7F861F73878CD360}{}
{
  
\section{\textcolor{Chapter }{Creating a polyhedra}}\label{Chapter_Functions_and_Methods_Section_Creating_a_polyhedra}
\logpage{[ 1, 1, 0 ]}
\hyperdef{L}{X7C86A67078A71BA7}{}
{
  

\subsection{\textcolor{Chapter }{Cdd{\textunderscore}PolyhedraByInequalities}}
\logpage{[ 1, 1, 1 ]}\nobreak
\hyperdef{L}{X7BEFA7A37C66FA20}{}
{\noindent\textcolor{FuncColor}{$\triangleright$\ \ \texttt{Cdd{\textunderscore}PolyhedraByInequalities({\mdseries\slshape arg})\index{CddPolyhedraByInequalities@\texttt{Cdd{\textunderscore}}\-\texttt{Polyhedra}\-\texttt{By}\-\texttt{Inequalities}}
\label{CddPolyhedraByInequalities}
}\hfill{\scriptsize (function)}}\\
\textbf{\indent Returns:\ }
a $\texttt{CddPolyhedra}$ Object 



 The function takes a list in which every entry represents an inequality( or
equality). In case we want some entries to represent equalities we should
refer in a second list to their indices. }

 
\begin{Verbatim}[commandchars=!@|,fontsize=\small,frame=single,label=Example]
  !gapprompt@gap>| !gapinput@A:= Cdd_PolyhedraByInequalities( [ [ 0, 1, 0 ], [ 0, 1, -1 ] ] );|
  < Polyhedra given by its H-representation >   
  !gapprompt@gap>| !gapinput@Display( A );|
  H-representation 
  Begin 
     2 X 3  rational
                
     0   1   0 
     0   1  -1 
  End
  !gapprompt@gap>| !gapinput@B:= Cdd_PolyhedraByInequalities( [ [ 0, 1, 0 ], [ 0, 1, -1 ] ], [ 2 ] );|
  < Polyhedra given by its H-representation >
  !gapprompt@gap>| !gapinput@Display( B );|
  H-representation 
  Linearity 1, [ 2 ]
  Begin 
     2 X 3  rational
                
     0   1   0 
     0   1  -1 
  End   
\end{Verbatim}
 

\subsection{\textcolor{Chapter }{Cdd{\textunderscore}PolyhedraByGenerators}}
\logpage{[ 1, 1, 2 ]}\nobreak
\hyperdef{L}{X7CBB524D81D0C0EE}{}
{\noindent\textcolor{FuncColor}{$\triangleright$\ \ \texttt{Cdd{\textunderscore}PolyhedraByGenerators({\mdseries\slshape arg})\index{CddPolyhedraByGenerators@\texttt{Cdd{\textunderscore}}\-\texttt{Polyhedra}\-\texttt{By}\-\texttt{Generators}}
\label{CddPolyhedraByGenerators}
}\hfill{\scriptsize (function)}}\\
\textbf{\indent Returns:\ }
a $\texttt{CddPolyhedra}$ Object 



 The function takes a list in which every entry represents a vertex in the
ambient vector space. In case we want some vertices to be free( the vertex and
its negative belong to the polyhedra) we should refer in a second list to
their indices . }

 
\begin{Verbatim}[commandchars=!@|,fontsize=\small,frame=single,label=Example]
  !gapprompt@gap>| !gapinput@A:= Cdd_PolyhedraByGenerators( [ [ 0, 1, 3 ], [ 1, 4, 5 ] ] );|
  < Polyhedra given by its V-representation >
  !gapprompt@gap>| !gapinput@Display( A );|
  V-representation 
  Begin 
     2 X 3  rational
              
     0  1  3 
     1  4  5 
  End
  !gapprompt@gap>| !gapinput@B:= Cdd_PolyhedraByGenerators( [ [ 0, 1, 3 ] ], [ 1 ] );      |
  < Polyhedra given by its V-representation >
  !gapprompt@gap>| !gapinput@Display( B );|
  V-representation 
  Linearity 1, [ 1 ]
  Begin 
     1 X 3  rational
              
     0  1  3 
  End
\end{Verbatim}
 }

 
\section{\textcolor{Chapter }{Some operations on polyhedras}}\label{Chapter_Functions_and_Methods_Section_Some_operations_on_polyhedras}
\logpage{[ 1, 2, 0 ]}
\hyperdef{L}{X7BF4D7BC8541CFCB}{}
{
  

\subsection{\textcolor{Chapter }{Cdd{\textunderscore}Canonicalize (for IsCddPolyhedra)}}
\logpage{[ 1, 2, 1 ]}\nobreak
\hyperdef{L}{X860134CC7A9A000B}{}
{\noindent\textcolor{FuncColor}{$\triangleright$\ \ \texttt{Cdd{\textunderscore}Canonicalize({\mdseries\slshape poly})\index{CddCanonicalize@\texttt{Cdd{\textunderscore}Canonicalize}!for IsCddPolyhedra}
\label{CddCanonicalize:for IsCddPolyhedra}
}\hfill{\scriptsize (operation)}}\\
\textbf{\indent Returns:\ }
a $\texttt{CddPolyhedra}$ Object 



 The function takes a polyhedra and reduces its defining inequalities (
generators set) by deleting all redundant inequalities ( generators ). }

 
\begin{Verbatim}[commandchars=!@|,fontsize=\small,frame=single,label=Example]
  !gapprompt@gap>| !gapinput@A:= Cdd_PolyhedraByInequalities( [ [ 0, 2, 6 ], [ 0, 1, 3 ], [1, 4, 10 ] ] );|
  < Polyhedra given by its H-representation >
  !gapprompt@gap>| !gapinput@B:= Cdd_Canonicalize( A );                                                       |
  < Polyhedra given by its H-representation >
  !gapprompt@gap>| !gapinput@Display( B );                                                             |
  H-representation 
  Begin 
     2 X 3  rational
                
     0   1   3 
     1   4  10 
  End
\end{Verbatim}
 

\subsection{\textcolor{Chapter }{Cdd{\textunderscore}V{\textunderscore}Rep (for IsCddPolyhedra)}}
\logpage{[ 1, 2, 2 ]}\nobreak
\hyperdef{L}{X7B1CDB8184EDF5C8}{}
{\noindent\textcolor{FuncColor}{$\triangleright$\ \ \texttt{Cdd{\textunderscore}V{\textunderscore}Rep({\mdseries\slshape poly})\index{CddVRep@\texttt{Cdd{\textunderscore}}\-\texttt{V{\textunderscore}Rep}!for IsCddPolyhedra}
\label{CddVRep:for IsCddPolyhedra}
}\hfill{\scriptsize (operation)}}\\
\textbf{\indent Returns:\ }
a $\texttt{CddPolyhedra}$ Object 



 The function takes a polyhedra and returns its reduced V-representation. }

 

\subsection{\textcolor{Chapter }{Cdd{\textunderscore}H{\textunderscore}Rep (for IsCddPolyhedra)}}
\logpage{[ 1, 2, 3 ]}\nobreak
\hyperdef{L}{X7AC96A9C7910F756}{}
{\noindent\textcolor{FuncColor}{$\triangleright$\ \ \texttt{Cdd{\textunderscore}H{\textunderscore}Rep({\mdseries\slshape poly})\index{CddHRep@\texttt{Cdd{\textunderscore}}\-\texttt{H{\textunderscore}Rep}!for IsCddPolyhedra}
\label{CddHRep:for IsCddPolyhedra}
}\hfill{\scriptsize (operation)}}\\
\textbf{\indent Returns:\ }
a $\texttt{CddPolyhedra}$ Object 



 The function takes a polyhedra and returns its reduced H-representation. }

 
\begin{Verbatim}[commandchars=!@|,fontsize=\small,frame=single,label=Example]
  !gapprompt@gap>| !gapinput@A:= Cdd_PolyhedraByInequalities( [ [ 0, 1, 1 ], [0, 5, 5 ] ] );                          |
   Polyhedra given by its H-representation >
  !gapprompt@gap>| !gapinput@B:= Cdd_V_Rep( A );                                    |
  < Polyhedra given by its V-representation >
  !gapprompt@gap>| !gapinput@Display( B );                                   |
  V-representation 
  Linearity 1, [ 2 ]
  Begin 
     2 X 3  rational
                
     0   1   0 
     0  -1   1 
  End
  !gapprompt@gap>| !gapinput@C:= Cdd_H_Rep( B );|
  < Polyhedra given by its H-representation >
  !gapprompt@gap>| !gapinput@Display( C );|
  H-representation 
  Begin 
     1 X 3  rational
              
     0  1  1 
  End
  !gapprompt@gap>| !gapinput@D:= Cdd_PolyhedraByInequalities( [ [ 0, 1, 1, 34, 22, 43 ], |
  !gapprompt@>| !gapinput@[ 11, 2, 2, 54, 53, 221 ], [33, 23, 45, 2, 40, 11 ] ] );|
  < Polyhedra given by its H-representation >
  !gapprompt@gap>| !gapinput@Cdd_V_Rep( C );                                                                                                     |
  < Polyhedra given by its V-representation >
  !gapprompt@gap>| !gapinput@Display( last );                                                                                                    |
  V-representation 
  Linearity 2, [ 5, 6 ]
  Begin 
     6 X 6  rational
                                                          
     1  -743/14   369/14    11/14        0        0 
     0    -1213      619       22        0        0 
     0       -1        1        0        0        0 
     0      764     -390      -11        0        0 
     0   -13526     6772       99      154        0 
     0  -116608    59496     1485        0      154 
  End
\end{Verbatim}
 }

 }

 \def\indexname{Index\logpage{[ "Ind", 0, 0 ]}
\hyperdef{L}{X83A0356F839C696F}{}
}

\cleardoublepage
\phantomsection
\addcontentsline{toc}{chapter}{Index}


\printindex

\newpage
\immediate\write\pagenrlog{["End"], \arabic{page}];}
\immediate\closeout\pagenrlog
\end{document}
