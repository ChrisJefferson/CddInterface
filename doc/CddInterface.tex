% generated by GAPDoc2LaTeX from XML source (Frank Luebeck)
\documentclass[a4paper,11pt]{report}

\usepackage{a4wide}
\sloppy
\pagestyle{myheadings}
\usepackage{amssymb}
\usepackage[utf8]{inputenc}
\usepackage{makeidx}
\makeindex
\usepackage{color}
\definecolor{FireBrick}{rgb}{0.5812,0.0074,0.0083}
\definecolor{RoyalBlue}{rgb}{0.0236,0.0894,0.6179}
\definecolor{RoyalGreen}{rgb}{0.0236,0.6179,0.0894}
\definecolor{RoyalRed}{rgb}{0.6179,0.0236,0.0894}
\definecolor{LightBlue}{rgb}{0.8544,0.9511,1.0000}
\definecolor{Black}{rgb}{0.0,0.0,0.0}

\definecolor{linkColor}{rgb}{0.0,0.0,0.554}
\definecolor{citeColor}{rgb}{0.0,0.0,0.554}
\definecolor{fileColor}{rgb}{0.0,0.0,0.554}
\definecolor{urlColor}{rgb}{0.0,0.0,0.554}
\definecolor{promptColor}{rgb}{0.0,0.0,0.589}
\definecolor{brkpromptColor}{rgb}{0.589,0.0,0.0}
\definecolor{gapinputColor}{rgb}{0.589,0.0,0.0}
\definecolor{gapoutputColor}{rgb}{0.0,0.0,0.0}

%%  for a long time these were red and blue by default,
%%  now black, but keep variables to overwrite
\definecolor{FuncColor}{rgb}{0.0,0.0,0.0}
%% strange name because of pdflatex bug:
\definecolor{Chapter }{rgb}{0.0,0.0,0.0}
\definecolor{DarkOlive}{rgb}{0.1047,0.2412,0.0064}


\usepackage{fancyvrb}

\usepackage{mathptmx,helvet}
\usepackage[T1]{fontenc}
\usepackage{textcomp}


\usepackage[
            pdftex=true,
            bookmarks=true,        
            a4paper=true,
            pdftitle={Written with GAPDoc},
            pdfcreator={LaTeX with hyperref package / GAPDoc},
            colorlinks=true,
            backref=page,
            breaklinks=true,
            linkcolor=linkColor,
            citecolor=citeColor,
            filecolor=fileColor,
            urlcolor=urlColor,
            pdfpagemode={UseNone}, 
           ]{hyperref}

\newcommand{\maintitlesize}{\fontsize{50}{55}\selectfont}

% write page numbers to a .pnr log file for online help
\newwrite\pagenrlog
\immediate\openout\pagenrlog =\jobname.pnr
\immediate\write\pagenrlog{PAGENRS := [}
\newcommand{\logpage}[1]{\protect\write\pagenrlog{#1, \thepage,}}
%% were never documented, give conflicts with some additional packages

\newcommand{\GAP}{\textsf{GAP}}

%% nicer description environments, allows long labels
\usepackage{enumitem}
\setdescription{style=nextline}

%% depth of toc
\setcounter{tocdepth}{1}





%% command for ColorPrompt style examples
\newcommand{\gapprompt}[1]{\color{promptColor}{\bfseries #1}}
\newcommand{\gapbrkprompt}[1]{\color{brkpromptColor}{\bfseries #1}}
\newcommand{\gapinput}[1]{\color{gapinputColor}{#1}}


\begin{document}

\logpage{[ 0, 0, 0 ]}
\begin{titlepage}
\mbox{}\vfill

\begin{center}{\maintitlesize \textbf{ CddInterface \mbox{}}}\\
\vfill

\hypersetup{pdftitle= CddInterface }
\markright{\scriptsize \mbox{}\hfill  CddInterface  \hfill\mbox{}}
{\Huge \textbf{ Gap interface to Cdd package \mbox{}}}\\
\vfill

{\Huge  0.1 \mbox{}}\\[1cm]
{ 18/11/2015 \mbox{}}\\[1cm]
\mbox{}\\[2cm]
{\Large \textbf{ Kamal Saleh\\
    \mbox{}}}\\
\hypersetup{pdfauthor= Kamal Saleh\\
    }
\end{center}\vfill

\mbox{}\\
{\mbox{}\\
\small \noindent \textbf{ Kamal Saleh\\
    }  Email: \href{mailto://kamal.saleh@rwth-aachen.de} {\texttt{kamal.saleh@rwth-aachen.de}}\\
  Homepage: \href{} {\texttt{}}\\
  Address: \begin{minipage}[t]{8cm}\noindent
 Templergraben \\
 \end{minipage}
}\\
\end{titlepage}

\newpage\setcounter{page}{2}
\newpage

\def\contentsname{Contents\logpage{[ 0, 0, 1 ]}}

\tableofcontents
\newpage

 \index{\textsf{CddInterface}}     
\chapter{\textcolor{Chapter }{Introduction}}\label{Chapter_Introduction}
\logpage{[ 1, 0, 0 ]}
\hyperdef{L}{X7DFB63A97E67C0A1}{}
{
  
\section{\textcolor{Chapter }{Why CddInterface}}\label{Chapter_Introduction_Section_Why_CddInterface}
\logpage{[ 1, 1, 0 ]}
\hyperdef{L}{X78B6806682EA4E0D}{}
{
  We know that every convex polyhedron has two representations, one as the
intersection of finite halfspaces and the other as Minkowski sum of the convex
hull of finite points and the nonnegative hull of finite directions. These are
called H-representation and V-representation, respectively. $\newline$ CddInterface is basicly written to translate between these two
representations. }

 
\section{\textcolor{Chapter }{H-representation and V-representation of polyhedra}}\label{Chapter_Introduction_Section_H-representation_and_V-representation_of_polyhedra}
\logpage{[ 1, 2, 0 ]}
\hyperdef{L}{X811322537C6ADBBE}{}
{
  Let us start by introducing the H-representation. Let $A$ be $m \times d$ matrix and let $b$ be a column $m$-vector. The H-representation of the polyhedron defined by the system $b+Ax \geq 0$ of $m$ inequalities and $d$ variables $x= (x_1,\dots,x_d)$ is as follows: $ \newline \textbf{H-representation}$ $\newline \textbf{linearity}\;t,\;\;[i_1, i_2, \dots,i_t]$ $\newline \textbf{begin}$ $\newline m \times (d+1)$ numbertype $\newline b \;\;\;\;\;\;\;\;\;\;\; A$ $\newline \textbf{end}$ $\newline$ The linearity line is added when we want to specify that some rows of the
system $b+Ax$ are equalities. That is, $k\in \{i_1, i_2, \dots,i_t\}$ means that the row $k$ of the system $b+Ax$ is specified to be equality. For example, the H-representation of the
polyhedron defined by the following system: $\newline 4-3x_1+6x_2-5x_4 = 0$ $\newline 1+2x_1-2x_2-7x_3 \geq 0$ $\newline -3x_2+5x_4 = 0\newline$ is as follows: $\newline \textbf{H-representation}$ $\newline \textbf{linearity}\;2,\;\;[1, 3]$ $\newline \textbf{begin}$ $\newline 3 \times 5$ rational $\newline 4\;\;\;-3\;\;\;\;\;\;\;\;\;6\;\;\;\;\;\;\;\;\;0\;\;\;\;-5$ $\newline 1\;\;\;\;\;\;\;\;2\;\;\;\;-2\;\;\;\;-7\;\;\;\;\;\;\;\;\;0$ $\newline 0\;\;\;\;\;\;\;\;0\;\;\;\;-3\;\;\;\;\;\;\;\;\;0\;\;\;\;\;\;\;\;\;5$ $\newline \textbf{end}$ $\newline \newline$ Next we define Polyhedra V-format. Let $P$ be represented by $n$ gerating points and $s$ generating directions (rays) as 
\[P = conv(v_1 , \dots , v_n ) + nonneg(r_{n+1} , \dots , r_{n+s} ).\]
 Then the Polyhedra V-format is for $P$ is: $\newline \newline \textbf{V-representation}$ $\newline \textbf{linearity}\;t,\;\;[i_1, i_2, \dots,i_t]$ $\newline \textbf{begin}$ $\newline (n+s) \times (d+1)$ numbertype $\newline 1 \;\;\;\;\;\;\;\;\;\;\; v_1$ $\newline \vdots \;\;\;\;\;\;\;\;\;\;\;\;\; \vdots$ $\newline 1 \;\;\;\;\;\;\;\;\;\;\; v_n$ $\newline 0 \;\;\;\;\;\;\;\;\;\;\; r_{n+1}$ $\newline \vdots \;\;\;\;\;\;\;\;\;\;\;\;\; \vdots$ $\newline 0 \;\;\;\;\;\;\;\;\;\;\; r_{n+s}$ $\newline \textbf{end}$ $\newline$ In the above format the generating points and generating rays may appear mixed
in arbitrary order. Linearity for V-representation specifies a subset of
generators whose coefficients are relaxed to be free. That is, $k \in \{i_1 , i_2 , . . . , i_t \}$ specifies that the $k$-th generator is specified to be free. This means for each such a ray $r_k$ , the line generated by $r_k$ is in the polyhedron, and for each such a vertex $v_k$ , its coefficient is no longer nonnegative but still the coefficients for all $v_i${\textquoteright}s must sum up to one. $\newline$ For example the V-representation of the polyhedron defined as 
\[P:= conv( (2,3), (-2,-3), (3,5) ) + nonneg(\; (1,2) , (-1,-2), (2,11)\;)\]
 $ \newline \textbf{V-representation}$ $\newline \textbf{linearity}\;2,\;\;[ 1,3 ]$ $\newline \textbf{begin}$ $\newline 4 \times 3$ numbertype $\newline 1 \;\;\;\;\;\;\;\; 2 \;\;\;\;\;\; 3 $ $\newline 1 \;\;\;\;\;\;\;\; 3 \;\;\;\;\;\; 5 $ $\newline 0 \;\;\;\;\;\;\;\; 1 \;\;\;\;\;\; 2 $ $\newline 0 \;\;\;\;\;\;\;\; 2 \;\;\;\;\;\; 11 $ $\newline \textbf{end}$ $\newline$ }

 }

   
\chapter{\textcolor{Chapter }{Creating polyhedras and their Operations}}\label{Chapter_Creating_polyhedras_and_their_Operations}
\logpage{[ 2, 0, 0 ]}
\hyperdef{L}{X80681BCE79BDBAC8}{}
{
  
\section{\textcolor{Chapter }{Creating a polyhedra}}\label{Chapter_Creating_polyhedras_and_their_Operations_Section_Creating_a_polyhedra}
\logpage{[ 2, 1, 0 ]}
\hyperdef{L}{X7C86A67078A71BA7}{}
{
  

\subsection{\textcolor{Chapter }{Cdd{\textunderscore}PolyhedraByInequalities}}
\logpage{[ 2, 1, 1 ]}\nobreak
\hyperdef{L}{X7BEFA7A37C66FA20}{}
{\noindent\textcolor{FuncColor}{$\triangleright$\ \ \texttt{Cdd{\textunderscore}PolyhedraByInequalities({\mdseries\slshape arg})\index{CddPolyhedraByInequalities@\texttt{Cdd{\textunderscore}}\-\texttt{Polyhedra}\-\texttt{By}\-\texttt{Inequalities}}
\label{CddPolyhedraByInequalities}
}\hfill{\scriptsize (function)}}\\
\textbf{\indent Returns:\ }
a $\texttt{CddPolyhedra}$ Object 



 The function takes a list in which every entry represents an inequality( or
equality). In case we want some entries to represent equalities we should
refer in a second list to their indices. }

 
\begin{Verbatim}[commandchars=!@|,fontsize=\small,frame=single,label=Example]
  !gapprompt@gap>| !gapinput@A:= Cdd_PolyhedraByInequalities( [ [ 0, 1, 0 ], [ 0, 1, -1 ] ] );|
  < Polyhedra given by its H-representation >   
  !gapprompt@gap>| !gapinput@Display( A );|
  H-representation 
  begin 
     2 X 3  rational
                
     0   1   0 
     0   1  -1 
  end
  !gapprompt@gap>| !gapinput@B:= Cdd_PolyhedraByInequalities( [ [ 0, 1, 0 ], [ 0, 1, -1 ] ], [ 2 ] );|
  < Polyhedra given by its H-representation >
  !gapprompt@gap>| !gapinput@Display( B );|
  H-representation 
  Linearity 1, [ 2 ]
  begin 
     2 X 3  rational
                
     0   1   0 
     0   1  -1 
  end   
\end{Verbatim}
 

\subsection{\textcolor{Chapter }{Cdd{\textunderscore}PolyhedraByGenerators}}
\logpage{[ 2, 1, 2 ]}\nobreak
\hyperdef{L}{X7CBB524D81D0C0EE}{}
{\noindent\textcolor{FuncColor}{$\triangleright$\ \ \texttt{Cdd{\textunderscore}PolyhedraByGenerators({\mdseries\slshape arg})\index{CddPolyhedraByGenerators@\texttt{Cdd{\textunderscore}}\-\texttt{Polyhedra}\-\texttt{By}\-\texttt{Generators}}
\label{CddPolyhedraByGenerators}
}\hfill{\scriptsize (function)}}\\
\textbf{\indent Returns:\ }
a $\texttt{CddPolyhedra}$ Object 



 The function takes a list in which every entry represents a vertex in the
ambient vector space. In case we want some vertices to be free( the vertex and
its negative belong to the polyhedra) we should refer in a second list to
their indices . }

 
\begin{Verbatim}[commandchars=!@|,fontsize=\small,frame=single,label=Example]
  !gapprompt@gap>| !gapinput@A:= Cdd_PolyhedraByGenerators( [ [ 0, 1, 3 ], [ 1, 4, 5 ] ] );|
  < Polyhedra given by its V-representation >
  !gapprompt@gap>| !gapinput@Display( A );|
  V-representation 
  begin 
     2 X 3  rational
              
     0  1  3 
     1  4  5 
  end
  !gapprompt@gap>| !gapinput@B:= Cdd_PolyhedraByGenerators( [ [ 0, 1, 3 ] ], [ 1 ] );      |
  < Polyhedra given by its V-representation >
  !gapprompt@gap>| !gapinput@Display( B );|
  V-representation 
  Linearity 1, [ 1 ]
  begin 
     1 X 3  rational
              
     0  1  3 
  end
\end{Verbatim}
 }

 
\section{\textcolor{Chapter }{Some operations on polyhedras}}\label{Chapter_Creating_polyhedras_and_their_Operations_Section_Some_operations_on_polyhedras}
\logpage{[ 2, 2, 0 ]}
\hyperdef{L}{X7BF4D7BC8541CFCB}{}
{
  

\subsection{\textcolor{Chapter }{Cdd{\textunderscore}Canonicalize (for IsCddPolyhedra)}}
\logpage{[ 2, 2, 1 ]}\nobreak
\hyperdef{L}{X860134CC7A9A000B}{}
{\noindent\textcolor{FuncColor}{$\triangleright$\ \ \texttt{Cdd{\textunderscore}Canonicalize({\mdseries\slshape poly})\index{CddCanonicalize@\texttt{Cdd{\textunderscore}Canonicalize}!for IsCddPolyhedra}
\label{CddCanonicalize:for IsCddPolyhedra}
}\hfill{\scriptsize (operation)}}\\
\textbf{\indent Returns:\ }
a $\texttt{CddPolyhedra}$ Object 



 The function takes a polyhedra and reduces its defining inequalities (
generators set) by deleting all redundant inequalities ( generators ). }

 
\begin{Verbatim}[commandchars=!@|,fontsize=\small,frame=single,label=Example]
  !gapprompt@gap>| !gapinput@A:= Cdd_PolyhedraByInequalities( [ [ 0, 2, 6 ], [ 0, 1, 3 ], [1, 4, 10 ] ] );|
  < Polyhedra given by its H-representation >
  !gapprompt@gap>| !gapinput@B:= Cdd_Canonicalize( A );                                                       |
  < Polyhedra given by its H-representation >
  !gapprompt@gap>| !gapinput@Display( B );                                                             |
  H-representation 
  begin 
     2 X 3  rational
                
     0   1   3 
     1   4  10 
  end
\end{Verbatim}
 

\subsection{\textcolor{Chapter }{Cdd{\textunderscore}V{\textunderscore}Rep (for IsCddPolyhedra)}}
\logpage{[ 2, 2, 2 ]}\nobreak
\hyperdef{L}{X7B1CDB8184EDF5C8}{}
{\noindent\textcolor{FuncColor}{$\triangleright$\ \ \texttt{Cdd{\textunderscore}V{\textunderscore}Rep({\mdseries\slshape poly})\index{CddVRep@\texttt{Cdd{\textunderscore}}\-\texttt{V{\textunderscore}Rep}!for IsCddPolyhedra}
\label{CddVRep:for IsCddPolyhedra}
}\hfill{\scriptsize (operation)}}\\
\textbf{\indent Returns:\ }
a $\texttt{CddPolyhedra}$ Object 



 The function takes a polyhedra and returns its reduced V-representation. }

 

\subsection{\textcolor{Chapter }{Cdd{\textunderscore}H{\textunderscore}Rep (for IsCddPolyhedra)}}
\logpage{[ 2, 2, 3 ]}\nobreak
\hyperdef{L}{X7AC96A9C7910F756}{}
{\noindent\textcolor{FuncColor}{$\triangleright$\ \ \texttt{Cdd{\textunderscore}H{\textunderscore}Rep({\mdseries\slshape poly})\index{CddHRep@\texttt{Cdd{\textunderscore}}\-\texttt{H{\textunderscore}Rep}!for IsCddPolyhedra}
\label{CddHRep:for IsCddPolyhedra}
}\hfill{\scriptsize (operation)}}\\
\textbf{\indent Returns:\ }
a $\texttt{CddPolyhedra}$ Object 



 The function takes a polyhedra and returns its reduced H-representation. }

 
\begin{Verbatim}[commandchars=!@|,fontsize=\small,frame=single,label=Example]
  !gapprompt@gap>| !gapinput@A:= Cdd_PolyhedraByInequalities( [ [ 0, 1, 1 ], [0, 5, 5 ] ] );                          |
   Polyhedra given by its H-representation >
  !gapprompt@gap>| !gapinput@B:= Cdd_V_Rep( A );                                    |
  < Polyhedra given by its V-representation >
  !gapprompt@gap>| !gapinput@Display( B );                                   |
  V-representation 
  Linearity 1, [ 2 ]
  begin 
     2 X 3  rational
                
     0   1   0 
     0  -1   1 
  end
  !gapprompt@gap>| !gapinput@C:= Cdd_H_Rep( B );|
  < Polyhedra given by its H-representation >
  !gapprompt@gap>| !gapinput@Display( C );|
  H-representation 
  begin 
     1 X 3  rational
              
     0  1  1 
  end
  !gapprompt@gap>| !gapinput@D:= Cdd_PolyhedraByInequalities( [ [ 0, 1, 1, 34, 22, 43 ], |
  !gapprompt@>| !gapinput@[ 11, 2, 2, 54, 53, 221 ], [33, 23, 45, 2, 40, 11 ] ] );|
  < Polyhedra given by its H-representation >
  !gapprompt@gap>| !gapinput@Cdd_V_Rep( D );                                                                                                     |
  < Polyhedra given by its V-representation >
  !gapprompt@gap>| !gapinput@Display( last );                                                                                                    |
  V-representation 
  Linearity 2, [ 5, 6 ]
  begin 
     6 X 6  rational
                                                          
     1  -743/14   369/14    11/14        0        0 
     0    -1213      619       22        0        0 
     0       -1        1        0        0        0 
     0      764     -390      -11        0        0 
     0   -13526     6772       99      154        0 
     0  -116608    59496     1485        0      154 
  end
\end{Verbatim}
 }

 }

   
\chapter{\textcolor{Chapter }{Linear Programs}}\label{Chapter_Linear_Programs}
\logpage{[ 3, 0, 0 ]}
\hyperdef{L}{X825271797BE64406}{}
{
  
\section{\textcolor{Chapter }{Creating a linear program}}\label{Chapter_Linear_Programs_Section_Creating_a_linear_program}
\logpage{[ 3, 1, 0 ]}
\hyperdef{L}{X8674A791858813B0}{}
{
  

\subsection{\textcolor{Chapter }{Cdd{\textunderscore}LinearProgram (for IsCddPolyhedra, IsString, IsList)}}
\logpage{[ 3, 1, 1 ]}\nobreak
\hyperdef{L}{X8186ADB179F58746}{}
{\noindent\textcolor{FuncColor}{$\triangleright$\ \ \texttt{Cdd{\textunderscore}LinearProgram({\mdseries\slshape poly, str, obj})\index{CddLinearProgram@\texttt{Cdd{\textunderscore}LinearProgram}!for IsCddPolyhedra, IsString, IsList}
\label{CddLinearProgram:for IsCddPolyhedra, IsString, IsList}
}\hfill{\scriptsize (operation)}}\\
\textbf{\indent Returns:\ }
a $\texttt{CddLinearProgram}$ Object 



 The function takes three variables. The first is a polyhedra $\texttt{poly}$, the second $\texttt{str}$ should be max or min and the third $\texttt{obj}$ is the objective. }

 

\subsection{\textcolor{Chapter }{Cdd{\textunderscore}SolveLinearProgram (for IsCddLinearProgram)}}
\logpage{[ 3, 1, 2 ]}\nobreak
\hyperdef{L}{X79EB266A8289CE29}{}
{\noindent\textcolor{FuncColor}{$\triangleright$\ \ \texttt{Cdd{\textunderscore}SolveLinearProgram({\mdseries\slshape lp})\index{CddSolveLinearProgram@\texttt{Cdd{\textunderscore}}\-\texttt{Solve}\-\texttt{Linear}\-\texttt{Program}!for IsCddLinearProgram}
\label{CddSolveLinearProgram:for IsCddLinearProgram}
}\hfill{\scriptsize (operation)}}\\
\textbf{\indent Returns:\ }
a list if the program is optimal, otherwise returns the value 0 



 The function takes a linear program. If the program is optimal, the function
returns a list of two entries: the solution vector and the optimal value of
the objective, otherwise it returns the value 0. }

 $\newline$ To illustrate the using of these functions, let us solve the linear program
given by: $\textbf{Maximize}\;\;\; P(x,y)= 1-2x+5y$, with $\newline$ $100\leq x \leq 200 \newline$ $80\leq y\leq 170 \newline$ $y \geq -x+200\newline\newline$ We bring the inequalities to the form $b+AX\geq 0$, we get: $\newline -100+x\geq 0 \newline$ $200-x \geq 0 \newline$ $-80+y \geq 0 \newline$ $170 -y \geq 0 \newline$ $-200 +x+y \geq 0 \newline$ 
\begin{Verbatim}[commandchars=!@|,fontsize=\small,frame=single,label=Example]
  !gapprompt@gap>| !gapinput@A:= Cdd_PolyhedraByInequalities( [ [ -100, 1, 0 ], [ 200, -1, 0 ], |
  !gapprompt@>| !gapinput@[ -80, 0, 1 ], [ 170, 0, -1 ], [ -200, 1, 1 ] ] );|
  < Polyhedra given by its H-representation >
  !gapprompt@gap>| !gapinput@Lp:= Cdd_LinearProgram( A, "max", [1, -2, 5 ] );|
  < Linear program >
  !gapprompt@gap>| !gapinput@S:= Cdd_SolveLinearProgram( Lp );|
  [ [ 100, 170 ], 651 ]
  !gapprompt@gap>| !gapinput@B:= Cdd_V_Rep( A );|
  < Polyhedra given by its V-representation >
  !gapprompt@gap>| !gapinput@Display( Lp );|
  Linear program given by: 
  H-representation 
  begin 
     5 X 3  rational
                      
     -100     1     0 
      200    -1     0 
      -80     0     1 
      170     0    -1 
     -200     1     1 
  end
  max  [ 1, -2, 5 ]
  !gapprompt@gap>| !gapinput@Display( B );|
  V-representation 
  begin 
     5 X 3  rational
                    
     1  100  170 
     1  100  100 
     1  120   80 
     1  200   80 
     1  200  170 
  end
\end{Verbatim}
 So the optimal solution is $(x=100,y=170)$ with optimal value $p=1-2(100)+5(170)=651$. }

 }

   
\chapter{\textcolor{Chapter }{Attributes and properties}}\label{Chapter_Attributes_and_properties}
\logpage{[ 4, 0, 0 ]}
\hyperdef{L}{X7DC480E57D26429A}{}
{
  
\section{\textcolor{Chapter }{Attributes and properties of polyhedra}}\label{Chapter_Attributes_and_properties_Section_Attributes_and_properties_of_polyhedra}
\logpage{[ 4, 1, 0 ]}
\hyperdef{L}{X83ED8E14812429F0}{}
{
  

\subsection{\textcolor{Chapter }{Cdd{\textunderscore}Dimension (for IsCddPolyhedra)}}
\logpage{[ 4, 1, 1 ]}\nobreak
\hyperdef{L}{X7ADD35A3858FB397}{}
{\noindent\textcolor{FuncColor}{$\triangleright$\ \ \texttt{Cdd{\textunderscore}Dimension({\mdseries\slshape poly})\index{CddDimension@\texttt{Cdd{\textunderscore}Dimension}!for IsCddPolyhedra}
\label{CddDimension:for IsCddPolyhedra}
}\hfill{\scriptsize (attribute)}}\\
\textbf{\indent Returns:\ }
The dimension of the polyhedra 



 }

 

\subsection{\textcolor{Chapter }{Cdd{\textunderscore}AmbientSpaceDimension (for IsCddPolyhedra)}}
\logpage{[ 4, 1, 2 ]}\nobreak
\hyperdef{L}{X877C7A557D0F0BC9}{}
{\noindent\textcolor{FuncColor}{$\triangleright$\ \ \texttt{Cdd{\textunderscore}AmbientSpaceDimension({\mdseries\slshape poly})\index{CddAmbientSpaceDimension@\texttt{Cdd{\textunderscore}}\-\texttt{Ambient}\-\texttt{Space}\-\texttt{Dimension}!for IsCddPolyhedra}
\label{CddAmbientSpaceDimension:for IsCddPolyhedra}
}\hfill{\scriptsize (attribute)}}\\
\textbf{\indent Returns:\ }
The dimension of the ambient space of the polyhedra 



 }

 

\subsection{\textcolor{Chapter }{Cdd{\textunderscore}GeneratingVertices (for IsCddPolyhedra)}}
\logpage{[ 4, 1, 3 ]}\nobreak
\hyperdef{L}{X8331AA15845EF762}{}
{\noindent\textcolor{FuncColor}{$\triangleright$\ \ \texttt{Cdd{\textunderscore}GeneratingVertices({\mdseries\slshape poly})\index{CddGeneratingVertices@\texttt{Cdd{\textunderscore}}\-\texttt{Generating}\-\texttt{Vertices}!for IsCddPolyhedra}
\label{CddGeneratingVertices:for IsCddPolyhedra}
}\hfill{\scriptsize (attribute)}}\\
\textbf{\indent Returns:\ }
The reduced generating vertices of the polyhedra 



 }

 

\subsection{\textcolor{Chapter }{Cdd{\textunderscore}GeneratingRays (for IsCddPolyhedra)}}
\logpage{[ 4, 1, 4 ]}\nobreak
\hyperdef{L}{X8358750A80380652}{}
{\noindent\textcolor{FuncColor}{$\triangleright$\ \ \texttt{Cdd{\textunderscore}GeneratingRays({\mdseries\slshape poly})\index{CddGeneratingRays@\texttt{Cdd{\textunderscore}GeneratingRays}!for IsCddPolyhedra}
\label{CddGeneratingRays:for IsCddPolyhedra}
}\hfill{\scriptsize (attribute)}}\\
\textbf{\indent Returns:\ }
The reduced generating rays of the polyhedra 



 }

 

\subsection{\textcolor{Chapter }{Cdd{\textunderscore}Equalities (for IsCddPolyhedra)}}
\logpage{[ 4, 1, 5 ]}\nobreak
\hyperdef{L}{X8443F98085882215}{}
{\noindent\textcolor{FuncColor}{$\triangleright$\ \ \texttt{Cdd{\textunderscore}Equalities({\mdseries\slshape poly})\index{CddEqualities@\texttt{Cdd{\textunderscore}Equalities}!for IsCddPolyhedra}
\label{CddEqualities:for IsCddPolyhedra}
}\hfill{\scriptsize (attribute)}}\\
\textbf{\indent Returns:\ }
The reduced defining equalities of the polyhedra 



 }

 

\subsection{\textcolor{Chapter }{Cdd{\textunderscore}Inequalities (for IsCddPolyhedra)}}
\logpage{[ 4, 1, 6 ]}\nobreak
\hyperdef{L}{X7B7EDB5379A9EBD5}{}
{\noindent\textcolor{FuncColor}{$\triangleright$\ \ \texttt{Cdd{\textunderscore}Inequalities({\mdseries\slshape poly})\index{CddInequalities@\texttt{Cdd{\textunderscore}Inequalities}!for IsCddPolyhedra}
\label{CddInequalities:for IsCddPolyhedra}
}\hfill{\scriptsize (attribute)}}\\
\textbf{\indent Returns:\ }
The reduced defining inequalities of the polyhedra 



 }

 

\subsection{\textcolor{Chapter }{Cdd{\textunderscore}IsEmpty (for IsCddPolyhedra)}}
\logpage{[ 4, 1, 7 ]}\nobreak
\hyperdef{L}{X7EA4A3E0818D071D}{}
{\noindent\textcolor{FuncColor}{$\triangleright$\ \ \texttt{Cdd{\textunderscore}IsEmpty({\mdseries\slshape poly})\index{CddIsEmpty@\texttt{Cdd{\textunderscore}IsEmpty}!for IsCddPolyhedra}
\label{CddIsEmpty:for IsCddPolyhedra}
}\hfill{\scriptsize (property)}}\\
\textbf{\indent Returns:\ }
$\texttt{true}$ if the polyhedra is empty and $\texttt{false}$ otherwise 



 }

 

\subsection{\textcolor{Chapter }{Cdd{\textunderscore}IsCone (for IsCddPolyhedra)}}
\logpage{[ 4, 1, 8 ]}\nobreak
\hyperdef{L}{X7998B5C280A2B07E}{}
{\noindent\textcolor{FuncColor}{$\triangleright$\ \ \texttt{Cdd{\textunderscore}IsCone({\mdseries\slshape poly})\index{CddIsCone@\texttt{Cdd{\textunderscore}IsCone}!for IsCddPolyhedra}
\label{CddIsCone:for IsCddPolyhedra}
}\hfill{\scriptsize (property)}}\\
\textbf{\indent Returns:\ }
$\texttt{true}$ if the polyhedra is cone and $\texttt{false}$ otherwise 



 }

 

\subsection{\textcolor{Chapter }{Cdd{\textunderscore}IsPointed (for IsCddPolyhedra)}}
\logpage{[ 4, 1, 9 ]}\nobreak
\hyperdef{L}{X80E8D896796D45DE}{}
{\noindent\textcolor{FuncColor}{$\triangleright$\ \ \texttt{Cdd{\textunderscore}IsPointed({\mdseries\slshape poly})\index{CddIsPointed@\texttt{Cdd{\textunderscore}IsPointed}!for IsCddPolyhedra}
\label{CddIsPointed:for IsCddPolyhedra}
}\hfill{\scriptsize (property)}}\\
\textbf{\indent Returns:\ }
$\texttt{true}$ if the polyhedra is pointed and $\texttt{false}$ otherwise 



 }

 }

 }

 \def\indexname{Index\logpage{[ "Ind", 0, 0 ]}
\hyperdef{L}{X83A0356F839C696F}{}
}

\cleardoublepage
\phantomsection
\addcontentsline{toc}{chapter}{Index}


\printindex

\newpage
\immediate\write\pagenrlog{["End"], \arabic{page}];}
\immediate\closeout\pagenrlog
\end{document}
